\documentclass[12pt]{article}
\usepackage[utf8]{inputenc}
\usepackage{amsmath, amssymb, graphicx, geometry}
\geometry{a4paper, margin=1in}
\usepackage{hyperref}
\hypersetup{
    colorlinks=true,
    linkcolor=blue,
    filecolor=magenta, 
    urlcolor=cyan,
}

\title{\textbf{Proyecto de Feria de Ciencias: Las Fases de la Luna}}
\author{Nombre del estudiante}
\date{\today}

\begin{document}

\maketitle

\tableofcontents

\newpage

\section{Introducción}
Las fases de la Luna son un fenómeno astronómico fascinante que resulta del movimiento relativo entre el Sol, la Tierra y la Luna. En este proyecto, exploraremos las causas de las fases lunares, realizaremos cálculos avanzados para predecir sus posiciones y diseñaremos un modelo físico y matemático para comprender mejor el proceso.

\section{Objetivo}
El objetivo de este proyecto es:
\begin{itemize}
    \item Explicar el fenómeno de las fases de la Luna desde un enfoque científico y matemático.
    \item Realizar cálculos para determinar las posiciones relativas de la Tierra, la Luna y el Sol en diferentes fases lunares.
    \item Crear un modelo visual o simulado para presentar en la feria de ciencias.
\end{itemize}

\section{Marco Teórico}
\subsection{Descripción de las Fases Lunares}
La Luna tiene ocho fases principales, que dependen de la fracción iluminada visible desde la Tierra:
\begin{itemize}
    \item Luna nueva
    \item Cuarto creciente
    \item Luna llena
    \item Cuarto menguante
\end{itemize}
Esto ocurre debido al ángulo formado entre el Sol, la Tierra y la Luna.

\subsection{Órbita de la Luna}
La órbita de la Luna es elíptica, con un semieje mayor promedio de 384,400 km. Su período orbital, conocido como mes sinódico, es de aproximadamente 29.53 días.

\section{Cálculos Matemáticos}
\subsection{Ángulo Sol-Tierra-Luna (\( \theta \))}
El ángulo \( \theta \) que determina la fase lunar se puede calcular como:
\[
\theta = 2\pi \frac{t}{T}
\]
donde:
\begin{itemize}
    \item \( t \): Tiempo transcurrido desde la última luna nueva (en días).
    \item \( T \): Período sinódico de la Luna (29.53 días).
\end{itemize}

\subsection{Fracción Iluminada (\( f \))}
La fracción iluminada visible desde la Tierra está dada por:
\[
 f = \frac{1}{2} \left( 1 + \cos(\theta) \right)
\]
\subsection{Distancia Tierra-Luna}
La distancia \( d \) se aproxima mediante:
\[
d = a (1 - e^2) / (1 + e \cos(\nu))
\]
donde:
\begin{itemize}
    \item \( a \): Semieje mayor (384,400 km).
    \item \( e \): Excentricidad de la órbita lunar (0.0549).
    \item \( \nu \): Anomalía verdadera de la órbita.
\end{itemize}

\section{Modelo Experimental}
\subsection{Materiales}
\begin{itemize}
    \item Una lámpara para representar el Sol.
    \item Una esfera para representar la Tierra.
    \item Otra esfera más pequeña para la Luna.
    \item Un soporte giratorio.
\end{itemize}

\subsection{Procedimiento}
\begin{enumerate}
    \item Coloca la lámpara en el centro del modelo.
    \item Sitúa la "Tierra" en un soporte giratorio a una distancia fija de la lámpara.
    \item Mueve la "Luna" alrededor de la "Tierra" mientras observas las fases desde un punto fijo en la "Tierra".
\end{enumerate}

\section{Resultados Esperados}
Se espera observar que las fases lunares son consecuencia directa de la posición relativa del Sol, la Tierra y la Luna. Los cálculos realizados permitirán predecir con precisión las fases y las distancias relativas.

\section{Conclusiones}
Este proyecto demuestra cómo las fases lunares son un fenómeno predecible basado en las leyes de la física y la geometría. La combinación de teoría, cálculos y modelos experimentales permite comprender mejor este fenómeno.

\section{Referencias}
\begin{enumerate}
    \item "Astronomy Today" de Chaisson y McMillan.
    \item "Introduction to Astronomy and Astrophysics" de Zeilik y Gregory.
    \item Recursos de la NASA: \url{https://moon.nasa.gov}
\end{enumerate}

\end{document}
